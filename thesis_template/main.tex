\documentclass[a4paper,12pt]{jreport}
% ,fleqn
\usepackage[dvipdfmx]{graphicx}
\usepackage{amsmath,txfonts}
\usepackage{theorem}
\usepackage{subfigure}
\usepackage{include/cite}
\usepackage{include/RSDLAB-thesis}
\renewcommand{\bibname}{参考文献}
%usepackage{./include/MeijoMaster}%%章ページにページ番号を表示する場合に読み込む
% \includeonly{chapters/experiment}
\usepackage{graphicx}
\graphicspath{
	{picture/information/}
	{picture/intro/}
	{picture/related/}
	{picture/system/}
	{picture/experiment/}
	{picture/conclusion/}
}
\usepackage{multirow}

%¥usepackage{comment}
%
%¥theoremstyle{break}

\setcounter{tocdepth}{3}
\setcounter{secnumdepth}{6}
\renewcommand{\thesubsubsection}{\arabic{chapter}.\arabic{section}.\arabic{subsection}.\arabic{subsubsection}}

\begin{document}
\pagestyle{fancy}
\thispagestyle{empty} %ノンブルと柱の設定
\begin{center}
	{\bf\Huge 卒 \hspace{12pt} 業 \hspace{12pt} 論 \hspace{12pt} 文 }\\ %\bfは太文字
	\vspace{2cm}
	{\Large 題目}\\
	\vspace{1cm}
	%論文タイトル
	{\bf\LARGE{サービスロボットの遠隔操作に対する\\リスクアセスメントに関する研究}}\\
	\vspace{2cm}
	指 導 教 員\\
	\vspace{0.5cm}
	{\Large \hspace{5pt} 大 原 \hspace{5pt} 賢 一 \hspace{5pt} 教 授  }\\
	\vspace{3cm}
	{\Large \hspace{5pt}220447701\\松野 \hspace{5pt} 有希}\\

	\vspace{1.3cm}
	{\Large 令和8年1月7日}\\
	{\Large 名城大学\\理工学部メカトロニクス工学科}
\end{center}
 %タイトル
\newpage %改ページ

\thispagestyle{empty}
\hspace{-4mm}\emph{{\LARGE 論文概要}}\\

近年, サービスロボットは人手不足や働き方の変化により, 社会での需要が高まっている. 一方, 設計段階においてリスクアセスメントが十分に考慮されていない場合, 運用時に想定外のリスクが発生して危険である. 本研究は, サービスロボットの設計時・運用時における安全性確保向上を目的として, リスクアセスメントを組み込んだ設計工程を提案する. 

本研究では, 提案した設計工程を自動化レベルの異なる4つのサービスロボットのユースケースに適用し, ダイアグラムを用いて危険源及びリスクの整理を行った. その結果, 自動化レベルの違いにより発生するリスクの種類や発生要因に差異が生じることが明らかとなった. また, 設計初期段階でリスクを明確化することで, 設計上のリスク低減策や運用ルールを

まだ、

検討しやすくなることが示された. ゆえに, 本研究で提案した設計工程は,サービスロボット導入時の安全性検討を支援する有効な手法であるといえる.  %概要

\clearpage %?
\pagenumbering{roman} %目次のノンブルの書体
\setcounter{page}{1} %目次のノンブルの番号の強制指定

\tableofcontents %目次の作成
\markright{目次} %2

\markright{図目次}
\listoffigures %図目次

\listoftables %表目次
\markright{表目次}
\newpage


\pagenumbering{arabic} %本文のノンブルの書体
\setcounter{page}{1} %本文のノンブルの番号の強制指定

%本文
\chapter{序論}
\label{chap:introduction}

\section{研究背景}
近年, サービスロボットは少子高齢化やパンデミックによる働き方の変化により, Fig. 1.1のような家庭内用掃除ロボットやFig. 1.2のような倉庫内運搬ロボット等, 様々な職業でロボットの普及が進んでいるといえる. 社会においては多様な役割を担い始めており, 特に遠隔操作型のサービスロボットは, 介護, 物流, 接客などの幅広い応用が期待されている. 実際に遠隔操作に関する先行研究も行われており, 将来的にはマニピュレータのある遠隔操作ロボットの需要は高まると考えられる.

\section{従来研究}
遠隔操作に関する研究では, 新規VRデバイスやソフトウェアに対応するためのシステム構築[]や操作性向上のためのUI設計[]等, 操作者の負担軽減や操作の直感性向上にに貢献したものが数多くある. しかし, こういったロボットを扱いやすくする研究だけでは社会実装するための安全面への配慮が不十分といえる. 安全性を確保するためにはリスクアセスメント(Risk Assessment, 以下RAとする)が重要である. 

RAとは, Jis規格[]やISO規格[]等の国際規格に基づく工学・安全工学のプロセスであり, リスクを完全に排除するのではなく, 許容可能な水準まで低減することを目的とした手法である. RAでは体系的にリスク低減を行うリスクアセスメントシートというツールがある. 

遠隔操作ロボットシステムに対して, RAを実施した研究としては, レストランの配膳ロボット[]や運搬ロボット[]を対象にRAを実施し, 安全性を検討した事例が報告されている. しかし, このような研究事例は未だ少なく, 多様なユースケースを想定した検討は十分に行われていない. その結果,遠隔操作ロボットシステムの設計・導入時に参照可能な知見が不足し,安全設計の考え方や実施段階が設計者個人に依存することで, 運用者が個別に適応する必要が生じ,さらに設計思想が共有されない場合には安全設計がブラックボックス化する.その結果,運用時に安全設計の意図や制約条件が十分に理解されないまま操作が行われ, 認知負荷が増大し,ヒューマンエラーが発生する可能性があるという課題がある.

\section{研究目的}
そこで本研究では, 遠隔操作のサービスロボットを対象に, 包括的なユースケースのもと, 個別事象ごとのリスクを整理し,設計工程における安全設計プロセスを体系的に示すことを目的とする. また, 包括的なユースケースにリスクアセスメントを実施することでロボット導入時に活用可能な知見やテンプレートの創出を目指す. 以下に, 本研究のアプローチを示す. 
\begin{enumerate}
  \item 既存の設計工程(SysMLを参考とした工程)にRAの工程を組み込んだ新たな設計工程の提案
  \item 自動化レベルごとに分類した4つのロボットユースケースに対する提案工程の適用
\end{enumerate}

\section{おわりに}
本章では,本研究を行う上での背景と課題を示した. 新たな安全設計プロセスを体系的に示すために, RAを組み込んだ新たな設計工程を示す.  %緒言
\chapter{提案設計工程の事前情報}

\section{はじめに}
本章では, 本研究で使用するRA, SysML, 自動化レベルについて説明する. また, 従来のロボットシステムの設計工程および従来のRA手法を整理し, 本研究の位置づけを確認する. 
\section{リスクアセスメントの概要}
リスクアセスメント[][]とは, JIS 規格や ISO 規格に基づき, 機械類に存在する危険源を体系的に洗い出し, それらによって生じるリスクを評価・低減するための規定である. RAの目的は, リスクを完全に排除することではなく, 許容可能な水準まで低減することで安全性を確保する点にある. RAは, 
\begin{enumerate}
  \item リスクシナリオの特定
  \item 危険源の特定
  \item リスクの評価
  \item リスク低減策の検討
\end{enumerate}
という手順で実施される. 具体的には, 機械類の使用目的や使用環境を考慮しながら, 危険源を同定し, それらが引き起こすリスクを評価する. その後, 各リスクに対して適切な低減策を検討・実施し, リスクが許容可能な水準に達していることを確認する. RAは, 機械類の設計段階から運用段階まで継続的に実施されるべきであり, 安全性の確保において重要な役割を果たす.

\subsection{リスクアセスメントシートの概要}
また, RAを実施するためのツールとしてリスクアセスメントシート(以下RAシートとする)がある. RAシートとは, RAを体系的に実施するための支援ツールであり, 設計者や導入者が危険源の特定からリスク評価, 低減策の検討までを一貫して行うことを目的として用いられる. 本研究では, サービスロボットを対象としたRAを実施するために, Excel形式のRAシートを用いる. RAシートは, 
\begin{enumerate}
  \item 表紙
  \item 初期分析・評価シート
  \item 方策後再分析シート
  \item 基本仕様
\end{enumerate}
以下に各シートを説明する.

\subsubsection{表紙}
表紙では主に, 意図した使用, 予見できる誤使用, 意図した空間/時間制限, リスク見積もりの計算方法を記載する. 
\begin{figure}[!h]
    \centering
    \includegraphics[width = 90mm]{cover.png}
    \caption{cover of RA sheet}
    \label{cover}
\end{figure}

今回使用したRAシートリスク値の計算では積算法(一部加算法)を使用する. この計算方法では, リスク値Rを, 危害のひどさSと, 危険にさらされる頻度F, 発生確率Ps, 回避可能性Aの和との積で表す. 
\begin{equation}
R = S \times (F + P_s + A)
\end{equation}
各指標は, RAシートに基づき定性的に評価し, その組み合わせからリスクレベルを算出する. 本研究では, リスクレベルが7以上の項目に対して低減策を検討する. 
\subsubsection{初期分析・評価シート}

\subsubsection{方策後再分析シート}

\subsubsection{基本仕様}

\subsection{SafeMLとは}

\section{ロボットシステムの設計工程}

\section{自動化レベル} %提案設計工程の事前情報
\chapter{RAを組み込んだ設計工程の提案}

\section{はじめに}
本章では, 本研究で提案するRAを組み込んだ設計工程と, その適用ユースケースについて説明する. 

\section{従来設計工程の課題}
SysMLは設計工程の進め方を規定するものではないが, システムエンジニアリングの観点から, 抽象的なシステム要件の定義から詳細な設計までを段階的に進めることが一般的である. 本研究では, 坂本氏らによるSysMLの解説書に示されているモデリングの流れを一例として採用し, これを従来設計工程として位置付ける. 設計工程は第2章2.3節で説明した順番である. 以下に, 順番を示す. 
\begin{figure}[!h]
    \centering
    \includegraphics[width = 85mm]{SysML.png}
    \caption{sysML}
    \label{method of sysml}
\end{figure}

しかし, 従来設計工程では, 各段階での安全設計に関する指針が明確に示されておらず, 設計者の経験や知識に依存する部分が大きい. そのため, 設計者によっては安全設計が十分に考慮されない場合があり, システムの安全性にばらつきが生じる可能性がある. また, 従来設計工程だけでは, 安全設計を体系的に進めるための具体的な手法やツールが不足している. これにより, 設計者は安全設計を効率的かつ効果的に進めることが難しく, 結果としてシステムの安全性が十分に確保されないリスクがある. 以上の課題を踏まえ, 本研究ではRAを組み込んだ新たな設計工程を提案することで, 安全設計を体系的かつ効率的に進めることを目指す.

\section{新規提案工程の概要}
本研究では, RAを組み込んだ設計工程として, SysMLの各図を用いた従来設計工程にRAシートとSafeMLの工程を追加し, 安全設計を体系的に進める手法を提案する. 以下に提案工程を示す. 
\begin{figure}[!h]
    \centering
    \includegraphics[width = 95mm]{newprocess.png}
    \caption{提案設計工程}
    \label{proposal_process}
\end{figure}

具体的には, SysMLのユースケース図とブロック定義図の間にRAシートによるリスク同定とリスク評価を行う工程, そこで低減策が必要となったリスク項目をSafeMLへ図としてモデル化する. そして, 必要となった低減策をSysMLの上工程に戻って安全機能を反映する. これにより, システム設計をする中で抽象的なシステム設計から具体的な構成要素を検討する間に規格に沿った安全設計をすることが出来る. つまり, 安全設計の視点からシステムの構造や振る舞いをより明確にし, 安全性の確保を効率的かつ効果的に進めることを目指す. 

\section{各工程の詳細}
ここでは, figで示した提案工程において, 新たに追加した箇所を説明する. 
\subsection{RAシートによるリスク同定・評価}
この工程では, 第2章2.1節で説明したRAシートを用いて, まず, 要求図, アクティビティ図, ユースケース図から意図した使用を検討する. 次に, アクティビティ図を参考にシステム全体の動作からリスクシナリオを抽出していく. さらに, 同定されたリスクにおいて, 2次被害が発生する可能性がある場合には, そのリスクシナリオも抽出していく. また, リスクの発生原因が複数ある場合は, 原因別にリスクシナリオを記述する. 本研究では, 人に与える物理的なリスクだけではなく, お店に対して信頼性を低下させるようなリスクも対象として同定する. これにより, システムの使用に伴う潜在的なリスクを網羅的に把握することができる. 続いて, RAシート表紙に記載した評価方法で定性的に各リスクのリスク値を算出する. そして, リスク値7以上となったリスク項目をSafeMLによるモデル化と低減策検討の対象とする.
\subsection{SafeMLによるリスクモデル化}
この工程では, 先に述べたRAシートによって対象となったリスク項目をSafeMLを用いてモデル化する. 本研究では, SafeMLに従ってRAシートのリスク項目を表形式からブロックと因果関係を表した線でモデル化する. 具体的には, RAシートの初期段階・評価シートに記載された項目を以下のように, 
\begin{itemize}
  \item 危険源:hazard block
  \item 危険事象/危険状態:context harm
  \item 危害:harm block
\end{itemize}
として表現する. これにより, RAシートの各要素の関係を視覚的に把握できるようになり, また, 重複する項目も整理されて, 低減策検討が効率化される. 低減策を検討した際には, RAシートの方策後再分析シートに低減策を記載して, リスク項目に対して効果があるのか, 閾値まで低減できているのかを評価する. さらに, 低減策が効果的でない場合には, 再度SafeMLモデルを見直し, 追加の低減策を検討する. これにより, RAシートによりリスクと低減策を評価し, SafeMLによりリスクの因果関係を視覚的に把握することが出来る. その結果, システムの安全性を効率的かつ効果的に確保することを目指す.

\subsection{安全機能をSysMLに反映}
この工程では, SafeMLで検討した低減策を設計者が必要なものを選び, SysMLの上工程に反映する. 具体的には, 要求図に安全要求として追加し, アクティビティ図でセンサによる回避動作を追加する. 操作者を支援するような外部に提供する機能があればユースケース図にも追加する. これにより, RA規格に沿ってシステム設計をすることが可能となる. また, システムの安全性を設計段階から考慮することができ, 結果として効率的にシステム全体の安全性が向上することが出来る. 

安全機能を反映した後は, ブロック定義図によりシステムの構成要素を検討する. 以降, 工程ごとにRAを繰り返すことで, システム全体の安全性を高める.  %提案設計工程の構築
\chapter{適用ユースケース}
ここでは構築するなにかしらのシステムを説明する.

\section{対象ユースケースの概要}

\section{自動化レベルの定義}

\section{各LOAのユースケース}
 %適用ユースケース
\chapter{提案工程の適用結果}
ここでは, 提案工程を各ユースケースに適用した結果について説明し, 考察を行う.
\section{ダイアグラム化結果}


\section{同定された危険源・リスク}


\section{設計時の注意点}


\section{考察}


 %提案工程の適用結果
\chapter{納言}
\section{本論文のまとめ}
\section{今後の課題}

 %結言
\chapter*{謝辞}

\addcontentsline{toc}{chapter}{謝辞}

本研究に携わる機会を与えていただき,終始親切なご指導をいただきました名城大学理工学部メカトロニクス工学科大原賢一教授に心からの感謝を申し上げます.

 %謝辞

%bibTeX(参考文献)の設定
\bibliography{sec/bib}
\bibliographystyle{jplain}
\end{document}
