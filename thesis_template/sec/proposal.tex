\chapter{RAを組み込んだ設計工程の提案}

\section{はじめに}
本章では, 本研究で提案するRAを組み込んだ設計工程と, その適用ユースケースについて説明する. 

\section{従来設計工程の課題}
SysMLは設計工程の進め方を規定するものではないが, システムエンジニアリングの観点から, 抽象的なシステム要件の定義から詳細な設計までを段階的に進めることが一般的である. 本研究では, 坂本氏らによるSysMLの解説書に示されているモデリングの流れを一例として採用し, これを従来設計工程として位置付ける. 設計工程は第2章2.3節で説明した順番である. 以下に, 順番を示す. 
\begin{figure}[!h]
    \centering
    \includegraphics[width = 85mm]{SysML.png}
    \caption{sysML}
    \label{method of sysml}
\end{figure}

しかし, 従来設計工程では, 各段階での安全設計に関する指針が明確に示されておらず, 設計者の経験や知識に依存する部分が大きい. そのため, 設計者によっては安全設計が十分に考慮されない場合があり, システムの安全性にばらつきが生じる可能性がある. また, 従来設計工程だけでは, 安全設計を体系的に進めるための具体的な手法やツールが不足している. これにより, 設計者は安全設計を効率的かつ効果的に進めることが難しく, 結果としてシステムの安全性が十分に確保されないリスクがある. 以上の課題を踏まえ, 本研究ではRAを組み込んだ新たな設計工程を提案することで, 安全設計を体系的かつ効率的に進めることを目指す.

\section{新規提案工程の概要}
本研究では, RAを組み込んだ設計工程として, SysMLの各図を用いた従来設計工程にRAシートとSafeMLの工程を追加し, 安全設計を体系的に進める手法を提案する. 以下に提案工程を示す. 
\begin{figure}[!h]
    \centering
    \includegraphics[width = 95mm]{newprocess.png}
    \caption{提案設計工程}
    \label{proposal_process}
\end{figure}

具体的には, SysMLのユースケース図とブロック定義図の間にRAシートによるリスク同定とリスク評価を行う工程, そこで低減策が必要となったリスク項目をSafeMLへ図としてモデル化する. そして, 必要となった低減策をSysMLの上工程に戻って安全機能を反映する. これにより, システム設計をする中で抽象的なシステム設計から具体的な構成要素を検討する間に規格に沿った安全設計をすることが出来る. つまり, 安全設計の視点からシステムの構造や振る舞いをより明確にし, 安全性の確保を効率的かつ効果的に進めることを目指す. 

\section{各工程の詳細}
ここでは, figで示した提案工程において, 新たに追加した箇所を説明する. 
\subsection{RAシートによるリスク同定・評価}
この工程では, 第2章2.1節で説明したRAシートを用いて, まず, 要求図, アクティビティ図, ユースケース図から意図した使用を検討する. 次に, アクティビティ図を参考にシステム全体の動作からリスクシナリオを抽出していく. さらに, 同定されたリスクにおいて, 2次被害が発生する可能性がある場合には, そのリスクシナリオも抽出していく. また, リスクの発生原因が複数ある場合は, 原因別にリスクシナリオを記述する. 本研究では, 人に与える物理的なリスクだけではなく, お店に対して信頼性を低下させるようなリスクも対象として同定する. これにより, システムの使用に伴う潜在的なリスクを網羅的に把握することができる. 続いて, RAシート表紙に記載した評価方法で定性的に各リスクのリスク値を算出する. そして, リスク値7以上となったリスク項目をSafeMLによるモデル化と低減策検討の対象とする.
\subsection{SafeMLによるリスクモデル化}
この工程では, 先に述べたRAシートによって対象となったリスク項目をSafeMLを用いてモデル化する. 本研究では, SafeMLに従ってRAシートのリスク項目を表形式からブロックと因果関係を表した線でモデル化する. 具体的には, RAシートの初期段階・評価シートに記載された項目を以下のように, 
\begin{itemize}
  \item 危険源:hazard block
  \item 危険事象/危険状態:context harm
  \item 危害:harm block
\end{itemize}
として表現する. これにより, RAシートの各要素の関係を視覚的に把握できるようになり, また, 重複する項目も整理されて, 低減策検討が効率化される. 低減策を検討した際には, RAシートの方策後再分析シートに低減策を記載して, リスク項目に対して効果があるのか, 閾値まで低減できているのかを評価する. さらに, 低減策が効果的でない場合には, 再度SafeMLモデルを見直し, 追加の低減策を検討する. これにより, RAシートによりリスクと低減策を評価し, SafeMLによりリスクの因果関係を視覚的に把握することが出来る. その結果, システムの安全性を効率的かつ効果的に確保することを目指す.

\subsection{安全機能をSysMLに反映}
この工程では, SafeMLで検討した低減策を設計者が必要なものを選び, SysMLの上工程に反映する. 具体的には, 要求図に安全要求として追加し, アクティビティ図でセンサによる回避動作を追加する. 操作者を支援するような外部に提供する機能があればユースケース図にも追加する. これにより, RA規格に沿ってシステム設計をすることが可能となる. また, システムの安全性を設計段階から考慮することができ, 結果として効率的にシステム全体の安全性が向上することが出来る. 

安全機能を反映した後は, ブロック定義図によりシステムの構成要素を検討する. 以降, 工程ごとにRAを繰り返すことで, システム全体の安全性を高める. 