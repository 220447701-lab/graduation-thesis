\chapter{序論}
\label{chap:introduction}

\section{研究背景}
近年, サービスロボットは少子高齢化やパンデミックによる働き方の変化により, Fig. 1.1のような家庭内用掃除ロボットやFig. 1.2のような倉庫内運搬ロボット等, 様々な職業でロボットの普及が進んでいるといえる. 社会においては多様な役割を担い始めており, 特に遠隔操作型のサービスロボットは, 介護, 物流, 接客などの幅広い応用が期待されている. 実際に遠隔操作に関する先行研究も行われており, 将来的にはマニピュレータのある遠隔操作ロボットの需要は高まると考えられる.

\section{従来研究}
遠隔操作に関する研究では, 新規VRデバイスやソフトウェアに対応するためのシステム構築[]や操作性向上のためのUI設計[]等, 操作者の負担軽減や操作の直感性向上にに貢献したものが数多くある. しかし, こういったロボットを扱いやすくする研究だけでは社会実装するための安全面への配慮が不十分といえる. 安全性を確保するためにはリスクアセスメント(Risk Assessment, 以下RAとする)が重要である. 

RAとは, Jis規格[]やISO規格[]等の国際規格に基づく工学・安全工学のプロセスであり, リスクを完全に排除するのではなく, 許容可能な水準まで低減することを目的とした手法である. RAでは体系的にリスク低減を行うリスクアセスメントシートというツールがある. 

遠隔操作ロボットシステムに対して, RAを実施した研究としては, レストランの配膳ロボット[]や運搬ロボット[]を対象にRAを実施し, 安全性を検討した事例が報告されている. しかし, このような研究事例は未だ少なく, 多様なユースケースを想定した検討は十分に行われていない. その結果,遠隔操作ロボットシステムの設計・導入時に参照可能な知見が不足し,安全設計の考え方や実施段階が設計者個人に依存することで, 運用者が個別に適応する必要が生じ,さらに設計思想が共有されない場合には安全設計がブラックボックス化する.その結果,運用時に安全設計の意図や制約条件が十分に理解されないまま操作が行われ, 認知負荷が増大し,ヒューマンエラーが発生する可能性があるという課題がある.

\section{研究目的}
そこで本研究では, 遠隔操作のサービスロボットを対象に, 包括的なユースケースのもと, 個別事象ごとのリスクを整理し,設計工程における安全設計プロセスを体系的に示すことを目的とする. また, 包括的なユースケースにリスクアセスメントを実施することでロボット導入時に活用可能な知見やテンプレートの創出を目指す. 以下に, 本研究のアプローチを示す. 
\begin{enumerate}
  \item 既存の設計工程(SysMLを参考とした工程)にRAの工程を組み込んだ新たな設計工程の提案
  \item 自動化レベルごとに分類した4つのロボットユースケースに対する提案工程の適用
\end{enumerate}

\section{おわりに}
本章では,本研究を行う上での背景と課題を示した. 新たな安全設計プロセスを体系的に示すために, RAを組み込んだ新たな設計工程を示す. 