\chapter{適用ユースケース}
本章では, 本研究で提案するRAを組み込んだ設計工程の適用ユースケースについて説明する. 今回対象とするユースケースは遠隔操作ロボットをサービスシステムで使用することを条件とする. 以下に, 自動化レベルの定義を説明し, 各自動化レベルに対応したユースケースについて説明する.

\section{自動化レベルの定義}
本研究では, 遠隔操作ロボットを対象とする. そこで, 2章4節で示した. LOAの表におけるMonitoringとGenerating, Implementingの項目に人が関与しているレベルと定義する. すると, レベル1からレベル4までが該当する. そこで, 本研究では, レベル1「Manual Control」, レベル2「Action Support」, レベル3「Batch Processing」, レベル4「Shared Control」の4つの自動化レベルに分類してユースケースを検討する.
\begin{figure}[!h]
    \centering
    \includegraphics[width = 150mm]{LOA4.png}
    \caption{4youthcase}
    \label{4youthcase}
\end{figure}

実際に, 遠隔操作ロボットがLOA1~4に該当するのか社会で使用されているロボットを例に調査した. まず, 遠隔操作ロボットの一例として, カフェ配膳ロボットのOriHimeを考える. 
\begin{figure}[!h]
    \centering
    \includegraphics[width = 85mm]{OriHime.png}
    \caption{OriHime}
    \label{OriHime}
\end{figure}
OriHimeは, 遠隔地からインターネットを通じて操作できるロボットであり, 主にカフェでの飲料配膳業務に利用されている. 操作者はタブレットやスマートフォンを使用して, OriHimeの動作をリアルタイムで制御することができる. 人の関与としては, 移動制御となり, コンピュータの関与は, 操作者のコマンド送信によるモーション動作となる. したがって, LOAの表に基づくと, OriHimeはレベル2「Action Support」に該当する. 

次に, 自律移動ロボットの一例として, レストランで配膳業務をするNyokkeyを考える.
\begin{figure}[!h]
    \centering
    \includegraphics[width = 85mm]{Nyokkey.png}
    \caption{Nyokkey}
    \label{Nyokkey}
\end{figure}
Nyokkeyは, レストラン内で自律的に移動し, 注文された料理や飲み物をテーブルまで配膳するロボットである. Nyokkeyは, リアルタイムで配膳物を認識してアームを使って配膳する. 人の関与としては, 配膳指示となり, コンピュータの関与は, 自律移動と障害物回避となる. したがって, LOAの表に基づくと, Nyokkeyはレベル6「Blended Decision Making」もしくはレベル7「Rigid System」に該当する.

ゆえに, 遠隔操作ロボットは自動化レベル4以下に該当すると考えられる. 

\section{各LOAのユースケース}
ここでは, 各自動化レベルに対応したユースケースについて説明する. 以下に, 各自動化レベルのユースケースを示す.
\subsection{LOA1: Manual Control}
LOA1「Manual Control」は, 遠隔操作ロボットが完全に人間の操作に依存している状態を指す. このレベルでは, ロボットは自律的な機能を持たず, すべての動作が人間の指示によって行われる. 例えば, 遠隔地にいる操作者が, ロボットの移動や作業をリアルタイムで制御する場合が該当する. このユースケースでは, 操作者がロボットの動作を細かく指示し, ロボットはその指示に従って動作する. したがって, ロボットの自律性はなく, すべての決定が人間によって行われる. 

LOA1のユースケースとして, Nyokkey(プロトタイプ)を使ったトイレ清掃とする. タスク内容としては, 遠隔地にいる操作者がNyokkeyを操作して, トイレ内の洗面台を小型モップで清掃を行う. 操作者は, ロボットから送信される映像と音声を見て, 移動やアーム操縦をリアルタイムで制御し, 施設内のトイレ清掃する. このユースケースでは, ロボットは自律的な機能を持たず, すべての動作が操作者の指示によって行われる. 以下に, 前提条件を示す. 
\begin{itemize}
  \item 人の役割:移動・アーム制御
  \item コンピュータの役割:自立判断なし
  \item 操作手段:専用コントローラ
  \item 使用環境:大学施設内および施設内トイレ(本学の建物を想定)
\end{itemize}

\subsection{LOA2: Action Support}
LOA2「Action Support」は, 遠隔操作ロボットが一部の機能において自律的な動作を行う状態を指す. このレベルでは, ロボットは特定のタスクや動作において自律的な機能を持ち, 人間の操作を補助する役割を果たす. 例えば, ロボットが自律的に障害物を回避しながら移動する場合や, 一部の作業を自動化する場合が該当する. このユースケースでは, ロボットは人間の指示に基づいて動作するが, 一部の動作はロボット自身が行う. したがって, ロボットの自律性は限定的であり, 人間と主導してロボットが補助する形でタスクを遂行する.

LOA2のユースケースとして, OriHimeを使ったカフェ配膳とする. タスク内容としては, 遠隔地にいる操作者がOriHimeを操作して, カフェ内で飲料を配膳する. 操作者は, ロボットから送信される映像と音声を見て, 移動やアーム操縦をリアルタイムで制御し, 顧客に飲料を提供する. 操作者は必要に応じてコマンド送信によるモーション動作をしたり, お客さんと話し接客をする. このユースケースでは, ロボットは一部の機能において自動的な動作を行い, 人間の操作を補助する役割を果たす. 以下に, 前提条件を示す.
\begin{itemize}
  \item 人の役割:首・移動(前後・回転)の操縦
  \item コンピュータの役割:コマンド送信によるモーション動作
  \item 操作手段:スマホ・タブレット
  \item 使用環境:カフェバーDawn(実店舗)
\end{itemize}

\subsection{LOA3: Batch Processing}
LOA3「Batch Processing」は, 遠隔操作ロボットが特定のタスクや動作を自律的に実行する状態を指す. このレベルでは, ロボットは一連のタスクや動作を自律的に遂行し, 人間の介入を最小限に抑えることができる. 例えば, ロボットが事前にプログラムされたルートに従って移動し, 一連の作業を自動的に実行する場合が該当する. このユースケースでは, ロボットは人間の指示に基づいて動作するが, ロボットの動作はロボット自身が行う. したがって, ロボットの自律性は高く, 人間の介入が必要な場面は限定的である.

このLOA3のユースケースとして, temiを使った施設見学とする. タスク内容としては, 遠隔地にいる見学者が, temiを通して, 施設を見学する. temiは自動追従機能を使って, 現地の案内人に追従しながら, 施設内を移動する. 見学者は, ロボットから送信される映像と音声を見て, 施設内の案内を受け, 様子を確認する. このユースケースでは, ロボットは特定のタスクや動作を自律的に実行し, 人間の介入を最小限に抑えることができる. 以下に, 前提条件を示す.
\begin{itemize}
  \item 人の役割:経路の生成・自動追従機能のプログラム
  \item コンピュータの役割:案内人追跡による自動移動
  \item 操作手段:なし(スマホ・タブレットからの映像視聴可能)
  \item 使用環境:工場施設内(本学の建物を想定)
\end{itemize}

\subsection{LOA4: Shared Control}
LOA4「Shared Control」は, 遠隔操作ロボット自律的に動作する中で人の操作が介入可能な状態を指す. このレベルでは, ロボットは自律的な機能を持ちつつ, 人間の操作を受け入れることができる. 例えば, ロボットが自律的に移動しながら, 緊急時は人間の指示を受け入れて動作を調整する場合が該当する. このユースケースでは, 基本的にロボットが自立制御を行いタスクを遂行する.