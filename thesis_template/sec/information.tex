\chapter{提案設計工程の事前情報}

\section{はじめに}
本章では, 本研究で使用するRA, SysML, 自動化レベルについて説明する. また, 従来のロボットシステムの設計工程および従来のRA手法を整理し, 本研究の位置づけを確認する. 
\section{リスクアセスメントの概要}
リスクアセスメント[][]とは, JIS 規格や ISO 規格に基づき, 機械類に存在する危険源を体系的に洗い出し, それらによって生じるリスクを評価・低減するための規定である. RAの目的は, リスクを完全に排除することではなく, 許容可能な水準まで低減することで安全性を確保する点にある. RAは, 
\begin{enumerate}
  \item リスクシナリオの特定
  \item 危険源の特定
  \item リスクの評価
  \item リスク低減策の検討
\end{enumerate}
という手順で実施される. 具体的には, 機械類の使用目的や使用環境を考慮しながら, 危険源を同定し, それらが引き起こすリスクを評価する. その後, 各リスクに対して適切な低減策を検討・実施し, リスクが許容可能な水準に達していることを確認する. RAは, 機械類の設計段階から運用段階まで継続的に実施されるべきであり, 安全性の確保において重要な役割を果たす.

\subsection{リスクアセスメントシートの概要}
また, RAを実施するためのツールとしてリスクアセスメントシート(以下RAシートとする)がある. RAシートとは, RAを体系的に実施するための支援ツールであり, 設計者や導入者が危険源の特定からリスク評価, 低減策の検討までを一貫して行うことを目的として用いられる. 本研究では, サービスロボットを対象としたRAを実施するために, Excel形式のRAシートを用いる. RAシートは, 
\begin{enumerate}
  \item 表紙
  \item 初期分析・評価シート
  \item 方策後再分析シート
  \item 基本仕様
\end{enumerate}
以下に各シートを説明する.

\subsubsection{表紙}
表紙では主に, 意図した使用, 予見できる誤使用, 意図した空間/時間制限, リスク見積もりの計算方法を記載する. 
\begin{figure}[!h]
    \centering
    \includegraphics[width = 90mm]{cover.png}
    \caption{cover of RA sheet}
    \label{cover}
\end{figure}

今回使用したRAシートリスク値の計算では積算法(一部加算法)を使用する. この計算方法では, リスク値Rを, 危害のひどさSと, 危険にさらされる頻度F, 発生確率Ps, 回避可能性Aの和との積で表す. 
\begin{equation}
R = S \times (F + P_s + A)
\end{equation}
各指標は, RAシートに基づき定性的に評価し, その組み合わせからリスクレベルを算出する. 本研究では, リスクレベルが7以上の項目に対して低減策を検討する. 
\subsubsection{初期分析・評価シート}

\subsubsection{方策後再分析シート}

\subsubsection{基本仕様}

\subsection{SafeMLとは}

\section{ロボットシステムの設計工程}

\section{自動化レベル}