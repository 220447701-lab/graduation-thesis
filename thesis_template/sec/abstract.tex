\hspace{-4mm}\emph{{\LARGE 論文概要}}\\

近年, サービスロボットは人手不足や働き方の変化により, 社会での需要が高まっている. 一方, 設計段階においてリスクアセスメントが十分に考慮されていない場合, 運用時に想定外のリスクが発生して危険である. 本研究は, サービスロボットの設計時・運用時における安全性確保向上を目的として, リスクアセスメントを組み込んだ設計工程を提案する. 

本研究では, 提案した設計工程を自動化レベルの異なる4つのサービスロボットのユースケースに適用し, ダイアグラムを用いて危険源及びリスクの整理を行った. その結果, 自動化レベルの違いにより発生するリスクの種類や発生要因に差異が生じることが明らかとなった. また, 設計初期段階でリスクを明確化することで, 設計上のリスク低減策や運用ルールを

まだ、

検討しやすくなることが示された. ゆえに, 本研究で提案した設計工程は,サービスロボット導入時の安全性検討を支援する有効な手法であるといえる. 